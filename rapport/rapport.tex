\documentclass[a4paper, 10pt, french]{article}
% Préambule; packages qui peuvent être utiles
   \RequirePackage[T1]{fontenc}        % Ce package pourrit les pdf...
   \RequirePackage{babel,indentfirst}  % Pour les césures correctes,
   % et pour indenter au début de chaque paragraphe
   \RequirePackage[utf8]{inputenc}   % Pour pouvoir utiliser directement les accents
   % et autres caractères français
   % \RequirePackage{lmodern,tgpagella} % Police de caractères
   \textwidth 17cm \textheight 25cm \oddsidemargin -0.24cm % Définition taille de la page
   \evensidemargin -1.24cm \topskip 0cm \headheight -1.5cm % Définition des marges
   % \RequirePackage{latexsym}                  % Symboles
   % \RequirePackage{amsmath}                   % Symboles mathématiques
   % \RequirePackage{tikz}   % Pour faire des schémas
   % \RequirePackage{graphicx} % Pour inclure des images
   % \RequirePackage{listings} % pour mettre des listings
   % Fin Préambule; package qui peuvent être utiles

\title{Rapport de TP 4MMAOD : Génération d'ABR optimal}
\author{
	Maxime Deloche \\
	Ludovic Carré \\
	\textit{Equipe Teide 4} \\
	\textbf{2ème partie du rendu}
}

\begin{document}

\maketitle

% \section{Equation de Bellman}
%
% \subsection{Question 1}
%
% \subsubsection{Montrons que tout sous-arbre d'un ABR optimal est un ABR optimal, par l'absurde.}
%
% Soit A un ABR optimal. On suppose qu'un sous-arbre S de A n'est pas optimal.
% Donc le nombre de comparaisons $x$ pour atteindre un élément $e$ de S n'est pas optimal dans S (notons cet optimal $x_{opt}$).
% Donc le nombre de comparaisons pour atteindre $e$ dans A est $x + 1$. Or, on pourrait atteindre $e$ en $x_{opt} + 1$, donc A n'est pas optimal, ce qui est absurde. \\
%
% Donc tout sous-arbre d'un ABR optimal est un ABR optimal.
%
%
% \subsubsection{Equation de Bellman}
%
% Deux raisonnements différents nous ont conduit à deux équations de Bellman différentes mais équivalentes. Nous les avons implémentées et on obtient des résultats identiques et corrects.\\
% \begin{itemize}
% 	\item[$\bullet$] \textbf{Version 1} \\
% 		Dans un arbre A de racine $e_i$, on a :
% 		\begin{itemize}
% 			\item[$\bullet$] une probabilité \verb?proba_que_element_soit_e_i? de faire 1 comparaison (si l'élément cherché est $e_i$).
% 			\item[$\bullet$] une probabilité \verb?proba_que_element_soit_dans_le_sous_arbre_gauche? de faire le nombre de comparaisons optimal du sous-arbre gauche, plus 1.
% 			\item[$\bullet$] une probabilité \verb?proba_que_element_soit_dans_le_sous_arbre_droit? de faire le nombre de comparaisons optimal du sous-arbre droit, plus 1.
% 		\end{itemize}
%
% 		Soit $c(E)$ le nombre de comparaisons moyen optimal de l'arbre formé par les éléments de E. L'équation de Bellman est : 
% 		\[ C(E) = \min_{e_i \in E} ( \frac{p_i}{\sum\limits_{e_l \in E}{p_l}} + \frac{C(\{e_j < e_i\}) + 1}{\sum\limits_{e_l \in E}{p_l}} \sum_{e_j < e_i}{p_j} + \frac{C(\{e_k > e_i\}) + 1}{\sum\limits_{e_l \in E}{p_l}} \sum_{e_k > e_i}{p_k} ) \]
% 		C'est-à-dire : 
% 		\[ C(E) = \min_{e_i \in E} ( 1 + \frac{C(\{e_j < e_i\})}{\sum\limits_{e_l \in E}{p_l}} \sum_{e_j < e_i}{p_j} + \frac{C(\{e_k > e_i\})}{\sum\limits_{e_l \in E}{p_l}} \sum_{e_k > e_i}{p_k} ) \]
% 		Conditions initiales : 
% 		\begin{itemize}
% 			\item $C(E) = 1$ si E ne contient qu'un élément (on ne fait alors qu'une comparaison).
% 			\item $C(E) = 0$ si E est vide. \\
% 		\end{itemize}
%
% 	\item[$\bullet$] \textbf{Version 2} \\
% 		En transformant la formule donnée de calcul du nombre de comparaison moyen optimal en version récursive : 
% 		Soit P la profondeur, initialisée à 1.
% 		L'équation de Bellman est : 
% 		\[ C(E, P) = \min_{e_i \in E} ( p_i.P + C(\{e_j < e_i\}, P+1) + C(\{e_k > e_i\}, P+1) )\]
% 		Conditions initiales : 
% 		\begin{itemize}
% 			\item $C(E) = p_i.P$ si $E = {e_i}$.
% 			\item $C(E) = 0$ si E est vide. \\
% 		\end{itemize}
%
% \end{itemize}
%
%
%
%
% \subsection{Question 2}
%
% \begin{itemize}
% 	\item[$\bullet$] \textbf{Coût de la version 1}
% 		\begin{itemize}
% 			\item \textbf{En mémoire} : on stocke dans un tableau de taille $n$ par $n$ le nombre moyen optimal de comparaisons, ainsi que le noeud choisi en racine pour cet arbre. 
% 				Par exemple, la position $(i, j)$ de ce tableau contiendra le nombre de comparaisons et la racine de l'abre formé par les éléments $e_i$ à $e_j$ 
% 				(en supposant les $e_i$ ordonnés). On a donc un coût en mémoire de $\theta(n^2)$. \\
% 			\item \textbf{En temps} : Pour chaque appel, on a de l'ordre de $k^2$ opérations à effectuer, avec $k$ le nombre d'éléments de E.
% 				On remarque qu'on a $(n - k)$ appels avec $k$ éléments, donc le coût est $\sum\limits_{k=0}^{n} (n-k)k^2 = \frac{1}{12}n(n+1)(n^2+3n+2)$.
% 				On a donc un coût en $\theta(n^4)$. \\
% 		\end{itemize}
% 	\item[$\bullet$] \textbf{Coût de la version 2}
% 		\begin{itemize}
% 			\item \textbf{En mémoire} : On a le même coût en mémoire qu'à la version 1 puisque la valeur optimale ne dépend pas de la profondeur. On est donc en $\theta(n^2)$.
% 			\item \textbf{En temps} : Pour un seul appel, on a un coût en $\theta(1)$ et puisque l'on a $n^2$ appels, on a un coût total de l'ordre de $\theta(n^2)$.\\
% 		\end{itemize}
% \end{itemize}
% \textbf{Conclusion} : On retient donc l'équation de la version 2 puisque elle propose une complexité plus avantageuse.
%
% \newpage
%

%%%%%%%%%%%%%%%%%%%%%%%%%%%%%%%%%%%%%%%%%%%%%%
\section{Principe de notre  programme (1 point)}
{Nous avons utilisé une méthode récursive pour le calcul du nombre moyen optimal de comparaisons (fonction \verb?avg_comp? de \verb?averagedepth.c?, appelée par \verb?get_avg?).

C'est l'implémentation directe de notre première version de la formule de Bellman, à laquelle nous avons ajouté des optimisations détaillées dans la partie suivante.

Une fois cette fonction appelée, la fonction \verb?build_tree? construit, à partir du tableau de mémoïsation, l'arbre optimal, et le retourne.

Enfin, la fonction \verb?treetoarray? écrit, à partir de cet arbre, le code C attendu sur la sortie standard (et peut au passage afficher la profondeur moyenne de cet arbre optimal, stockée dans l'objet \verb?Tree?).

Nous avons manqué de temps pour pousser l'implémentation : en effet, nous avions commencé par une implémentation en Python. Nous nous sommes cependant vite rendus compte de la difficulté à effectuer le moindre calcul d'optimisation et de la lenteur du langage, et avons donc décidé de passer à du C.
} 

%%%%%%%%%%%%%%%%%%%%%%%%%%%%%%%%%%%%%%%%%%%%%%
\section{Analyse du coût théorique (2 points)}
{\em Donner ici l'analyse du coût théorique de votre programme en fonction du nombre $n$ d'éléments dans le dictionnaire.
 Pour chaque coût, donner la formule qui le caractérise (en justifiant brièvement pourquoi cette formule correspond à votre programme), 
 puis l'ordre du coût en fonction de $n$ en notation $\Theta$ de préférence, sinon $O$.}

 \subsection{Choix d'implémentations}
 \begin{itemize}
	 \item Le tableau \verb?comp_array? est un tableau de mémoïsation : à la position \verb?(i, j)?, il stocke l'indice de la racine choisie pour le sous-arbre formé des noeuds d'indices \verb?i? (inclus) à \verb?j? (exclus), et le nombre de comparaisons moyen de cet arbre optimal.

		 Nous n'avions pas besoin de stocker l'arbre en entier, seulement la racine : en effet, si on sait que la racine est d'indice \verb?k?, on trouve son fils gauche (resp. droit) dans la case \verb?(i, k)? (resp. \verb?(k+1, j)?).

		 C'est ce tableau qui est parcouru par la fonction \verb?build_tree? pour reconstituer l'arbre optimal.

	 \item Nous avons à chaque appel récursif besoin de sommer les probabilités de \verb?i? à \verb?j? par exemple. A la lecture du fichier de données, nous remplissons un tableau \verb?proba_sums? stockant les sommes des probas des indices 0 à \verb?i?.
		 Nous accédons donc ensuite à la somme de probabilités de \verb?i? à \verb?j? en calculant \verb?proba_sums[j] - proba_sums[i]?, donc en coût constant.
 \end{itemize}

  \subsection{Nombre  d'opérations en pire cas \em TODO}
	\paragraph{Justification\,: }
	{\em La justification peut être par exemple de la forme: \\ 
	   "Le programme itératif contient les boucles $k_1=...$, $k_2= ...$ etc correspondant à la somme 
	  $$T(n_1, n_2, c_1, c_2) = \sum_{k_1=...}^{...} ... \sum ... + \sum_{i=...}^{...} ...$$ 
	  somme que nous avons calculée (ou majorée) par la technique de  ... " \\
	  ou  encore\,:  \\
	  "les appels récursifs du programme permettent de modéliser son coût par le système d'équations aux récurrences 
	  $$T(k_1, k_2) = ...  \mbox{~avec~les~conditions~initiales~....~} $$
	  Le coût indiqué est obtenu en résolvant ce système par la méthode de  .... "
	} 
  \subsection{Place mémoire requise}
  La place mémoire requise est en $O(n^2)$, avec $n$ le nombre d'éléments de l'arbre.
	\paragraph{Justification}
	La plus grande place mémoire nécessaire vient du tableau de mémoïsation, qui stocke deux valeurs pour chaque couple d'indices, et a donc un coût de l'ordre de $n^2$ (en pratique 2 fois moins, puisque l'on ne stocke quelque chose que lorsque $low\_index < high\_index$).
	Les tableaux de probabilités coûtent une taille $n$ en mémoire, de même que l'arbre, ce qui est donc négligeable devant $n^2$.

  \subsection{Nombre de défauts de cache sur le modèle CO \em TODO}
	\paragraph{Justification}
	Bof

	%%%%%%%%%%%%%%%%%%%%%%%%%%%%%%%%%%%%%%%%%%%%%%
\section{Compte rendu d'expérimentation (2 points)}
  \subsection{Conditions expérimentaless}
	 {\em Décrire les conditions permettant la reproductibilité des mesures: on demande la description
	  de la machine et la méthode utilisée pour mesurer le temps.
	 }

	\subsubsection{Description synthétique de la machine\,:} 
	  {\em indiquer ici le  processeur et sa fréquence, la mémoire, le système d'exploitation. 
	   Préciser aussi si la machine était monopolisée pour un test, ou notamment si 
	   d'autres processus ou utilisateurs étaient en cours d'exécution. 
	  } 

	\subsubsection{Méthode utilisée pour les mesures de temps\,: } 
	  {\em préciser ici  comment les mesures de temps ont été effectuées (fonction appelée) et l'unité de temps; en particulier, 
	   préciser comment les 5 exécutions pour chaque test ont été faites (par exemple si le même test est fait 5 fois de suite, ou si les tests sont alternés entre
	   les mesures, ou exécutés en concurrence etc). 
	  }

  \subsection{Mesures expérimentales}
	{\em Compléter le tableau suivant par les temps d'exécution mesurés pour chacun des 6 benchmarks imposés
			  (temps minimum, maximum et moyen sur 5 exécutions)
	}\\
	{Les mesures ont été effectuées sur la fonction itérative \verb?avg_comp_iter? avant l'optimisation des défauts de cache.}

	\begin{figure}[h]
		\begin{center}
			\begin{tabular}{|l||r||r|r|r||}
				\hline
				\hline
				& Taille         & temps     & temps   & temps \\
				& du test     & min       & max     & moyen \\
				\hline
				\hline
				benchmark1 & 5   & 0.002     & 0.005   & 0.015 \\
				\hline
				benchmark2 & 10  & 0.002     & 0.003   & 0.002 \\
				\hline
				benchmark3 & 1000  & 0.029     & 0.061   & 0.044 \\
				\hline
				benchmark4 & 2000  & 0.128    & 0.155    & 0.136 \\
				\hline
				benchmark5 & 3000  & 0.279    & 0.321    & 0.302 \\
				\hline
				benchmark6 & 5000  & 0.890    & 1.011    & 0.937 \\
				\hline
				\hline
			\end{tabular}
			\caption{Mesures des temps minimum, maximum et moyen de 5 exécutions pour les 6 benchmarks.}
			\label{table-temps}
		\end{center}
	\end{figure}

\subsection{Analyse des résultats expérimentaux}
{\em Donner  une réponse justifiée  à la question\,: 
			  les  temps mesurés correspondent ils  à votre analyse théorique (nombre d’opérations et défauts de cache) ?
}

\end{document}
%% Fin mise au format

